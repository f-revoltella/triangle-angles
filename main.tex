\documentclass{article}

\usepackage{amsmath}

\title{Why the Angles in a Triangle Add Up to 180 Degrees}
\author{}
\date{}

\begin{document}

\maketitle

\section{Introduction}

A triangle is a polygon with three sides and three interior angles.
In this document, we show why the sum of the interior angles of any triangle is
\(180^\circ\).

\section{Geometric Argument}

Consider a triangle with interior angles labeled \(A\), \(B\), and \(C\).

\subsection*{Steps of the proof}

\begin{enumerate}
    \item Draw a triangle and extend one of its sides to form a straight line.

    \item Through the opposite vertex, draw a line parallel to the base of the triangle.

    \item By the properties of parallel lines, the alternate interior angles formed
    are equal to angles \(A\) and \(B\).

    \item These two angles and angle \(C\) lie on a straight line, which measures
    \(180^\circ\).

    \item Therefore, the sum of the angles in the triangle is:
    \[
    A + B + C = 180^\circ
    \]
\end{enumerate}

\section{Conclusion}

This argument shows that the interior angles of any triangle always add up to
\(180^\circ\), regardless of the triangle’s shape.

\end{document}
